\documentclass[a4paper]{scrreprt}

\usepackage{scrhack}
\usepackage{graphicx}
\usepackage[utf8]{inputenc}

\addtokomafont{titlehead}{\flushright}
\addtokomafont{subject}{\vspace{3cm}\flushleft}
\addtokomafont{title}{\flushleft}
\addtokomafont{subtitle}{\flushleft}
\addtokomafont{author}{\flushleft\setlength{\tabcolsep}{0pt}}
\addtokomafont{date}{\flushleft}
\addtokomafont{publishers}{\flushleft}

\titlehead{\includegraphics[scale=2]{../templates/logo_en}}
\subject{Software Engineering and Design}
\title{Requirements Specification}
\subtitle{Mental Health Care Patient Management System (MHC-PMS)}
\author{
\begin{tabular}{l}
\normalfont\bfseries{Team White:}\\
Dellsperger Jan\\
Ellenberger Roger\\
Sheppard David\\
Sidler Matthias\\
Spring Mathias\\
Thöni Stefan
\end{tabular}
}
\date{\today}
\publishers{Version 1.0}

\begin{document}

\begin{titlepage}
	\maketitle
\end{titlepage}


\tableofcontents


\chapter{Preface}

\chapter{Introduction}

\chapter{Glossary}

\chapter{User requirements definition}
% Here, you describe the services provided for the user. The nonfunctional system requirements should also be described in this section. This description may use natural language, diagrams, or other notations that are understandable to customers. Product and process standards that must be followed should be specified. (Use Case- and Activity Diagrams)

\chapter{System architecture}
% This chapter should present a high-level overview of the anticipated system architecture, showing the distribution of functions across system modules. Architectural components that are reused should be highlighted.

\chapter{System requirements specification}
% This should describe the functional and nonfunctional requirements in more detail. If necessary, further detail may also be added to the nonfunctional requirements. Interfaces to other systems may be defined.

\chapter{System models}
% This might include graphical system models showing the relationships between the system components and the system and its environment. Examples of possible models are object models, data- flow,models or semantic data models.


\chapter{System evolution}
% This should describe the fundamental assumptions on which the system is based, and any anticipated changes due to hardware evolution, changing user needs, and so on. This section is useful for system designers as it may help them avoid design decisions that would constrain likely future changes to the system.
In zukünftigen Versionen könnten folgende erweiternde oder neue Funktionen implementiert werden:
\begin{itemize}
\item Weitere Reports\\
Aufgrund von Rückmeldungen und Wünschen aus dem Daily Business können weitere Reports definiert werden.
\item Eigene Reports definieren\\
Der User kann direkt in der Applikation auf den vorhandenen Daten eigene Reports konfigurieren und abspeichern.
\end{itemize}

\chapter{Testing}
% This chapter should contain an overview description of what and how you want to test your system.

\chapter{Appendix}

\chapter{Index}






\end{document}