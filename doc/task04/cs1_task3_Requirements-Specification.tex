\documentclass[a4paper]{scrreprt}

\usepackage{scrhack}
\usepackage{graphicx}
\usepackage[utf8]{inputenc}

\addtokomafont{titlehead}{\flushright}
\addtokomafont{subject}{\vspace{3cm}\flushleft}
\addtokomafont{title}{\flushleft}
\addtokomafont{subtitle}{\flushleft}
\addtokomafont{author}{\flushleft\setlength{\tabcolsep}{0pt}}
\addtokomafont{date}{\flushleft}
\addtokomafont{publishers}{\flushleft}

\titlehead{\includegraphics[scale=2]{../templates/logo_en}}
\subject{Software Engineering and Design}
\title{Requirements Specification}
\subtitle{Mental Health Care Patient Management System (MHC-PMS)}
\author{
\begin{tabular}{l}
\normalfont\bfseries{Team White:}\\
Dellsperger Jan\\
Ellenberger Roger\\
Sheppard David\\
Sidler Matthias\\
Spring Mathias\\
Thöni Stefan
\end{tabular}
}
\date{\today}
\publishers{Version 1.0}

\begin{document}

\begin{titlepage}
	\maketitle
\end{titlepage}


\tableofcontents


\chapter{Vorwort}
% This should define the expected readership of the document and describe its version history, including a rationale for the creation of a new version and a summary of the changes made in each version.


\section{Über dieses Dokument}
Dieses Dokument beschreibt den Requirements-Engineering-Prozess des Projekts \textit{MHC-PMS}. Es spezifiziert die Erkenntnisse auf dem Design-Thinking-Prozess.


\section{Zielgruppe}
Das Dokument richtet sich an den Endkunden, die Projektleitung, die Personalplanung, die Entwickler, Test-Engineers und das zukünftige Betriebsteam.


\section{Änderungsnachweis}
\begin{table}[h]
\label{tab_version-history}
\begin{tabular}{llll}
{\bf Version} & {\bf Beschreibung} 							& {\bf Autor} 	& {\bf Datum} \\
0.1.0         & Dokument aus Vorlage erstellt 				& eller1 		& 01.04.2016  \\
0.2.0         & Vorwort verfasst und Glossar erstellt		& eller1 		& 01.04.2016  \\
0.3.0         & Einleitung verfasst			                & eller1 		& 01.04.2016  \\

\end{tabular}
\end{table}



\chapter{Einleitung}
% This should describe the need for the system. It should briefly describe the system’s functions and explain how it will work with other systems. It should also describe how the system fits into the overall business or strategic objectives of the organization commissioning the software

Die Betreuung von Personen mit psychischen Störungen soll mit unserer Software vereinfacht werden. Die Zielkundschaft sind kleine bis grosse Einrichtungen für die Behandlung (ambulant und Hausbesuche) von Patienten mit psychischen Störungen. Wir fokussieren dabei auf Funktionen für das Management. Wir möchten Personen mit Führungsfunktion bei Planungsarbeiten, administrativen Tätigkeiten und Strategie-Entscheiden unterstützen. Die Verwaltung und Auswertung von Patientendaten (Behandlungshistorie und Verrechnung) steht dabei im Fokus. Zudem soll die Personalplanung mit der Patientenverwaltung verknüpf werden. Export von Berichten für Partnerorganisationen und Behörden soll möglichst unkompliziert gestaltet werden.

\bigskip

Unser Produkt ist rein für die Datenauswertung gedacht. Als Datenquellen nutzen wir eine bestehende Patientenverwaltung (mit Personalplanung und Verrechnung). Durch die gewonnen Erkenntnisse soll das Management ihre Strategie mit Fakten fundiert steuern können und im Alltag weniger administrativen Aufwand vorfinden. In er Gesundheitsbranche sollen damit Kosteneinsparungen erreicht werden. Ineffiziente Behandlungsarten können schneller erkannt und Patienten mit schlechten Behandlungserfolg besser überwacht werden.



\chapter{Glossar}
\begin{table}[h]
\label{tab_glossar}
\begin{tabular}{llll}
{\bf Begriff} 		& {\bf Beschreibung} \\
Dashboard			& Übersichtseite zu einem Themenbereich \\

HIT 				& Health IT: Informatik-Sparte, die sich mit dem Gesundheitswesen auseinandersetzt \\

MHC-PMS 			& Mental Health Care Patient Management System \\
Report				& Bericht


\end{tabular}
\end{table}




\chapter{User-Requirements Definition}
% Here, you describe the services provided for the user. The nonfunctional system requirements should also be described in this section. This description may use natural language, diagrams, or other notations that are understandable to customers. Product and process standards that must be followed should be specified. (Use Case- and Activity Diagrams)


\section{User Requirements}



\section{Use-Cases}



\section{Aktivitätsdiagramm}


\section{Szenarien}





\chapter{System Architektur}




\chapter{System-Requirements Spezifikation}

\section{Functional Requirements}
\subsection{UC1: Dashboard}
\paragraph{[1] Darstellung Dashboard}
Das System soll beim starten oder alternativ per Klick auf einen Menu-Punkt ein Dashboard anzeigen. Dieses Dashboard soll, in Kacheln dargestellt, konfigurierte Kennzahlen darstellen. (Siehe Requirement [4]) Desweiteren soll per Klick auf einen solchen Kachel der entsprechende Report angezeigt werden. 
	
\paragraph{[2] Dashboard Konfiguration}
Das System soll einen Menu-Punkt zur verfügung stellen der die Konfiguration des Dashboards erlaubt. 
\paragraph{[3] Schnittstelle DBS}
\paragraph{[4] Kennzahlen}

\section{Non-Functional Requirements}
\paragraph{Performance: Starten der Applikation}
Das System soll nicht länger als 4 Sekunden benötigen um nach Start der Applikation das Dashboard anzuzeigen. (4 Sekunden sollten genügen um die relevanten Daten zu laden und anzuzeigen. Ausserdem ist es genügend kurz um den Benutzer nicht zu belasten)
\footnote{http://www.hobo-web.co.uk/your-website-design-should-load-in-4-seconds/}



\section{Domain Requirements}
\subsection{Branchenspezifisch}
\paragraph{Leistungsabrechnung} Implementieren der Standards für Leistungserfassung gemäss TARMED.

\paragraph{Meldepflicht} Ermöglichen von Exports der Behandlungshistorie aus den Patientendaten.

\paragraph{Arztgeheimnis} Gewährleisten vom Arztgeheimnis durch Implementation von Berechtigungsstrukturen auf Benutzerebene.


\subsection{Rechtlich}
\paragraph{Datenschutz} Einhalten der schweizerischen Datenschutzrichtlinien. 

\paragraph{Aufbewahrungspflicht} Einhalten der 10-Jährigen Datenaufbewahrungspflicht.




\chapter{System Models}
% This might include graphical system models showing the relationships between the system components and the system and its environment. Examples of possible models are object models, data- flow , models or semantic data models.



\chapter{System Evolution}



\chapter{Testing}



\chapter{Appendix}





\chapter{Index}






\end{document}