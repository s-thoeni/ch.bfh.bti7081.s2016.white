\documentclass[a4paper]{scrartcl}

\usepackage{scrhack}
\usepackage{graphicx}
\usepackage[utf8]{inputenc}

\addtokomafont{titlehead}{\flushright}
\addtokomafont{subject}{\vspace{3cm}\flushleft}
\addtokomafont{title}{\flushleft}
\addtokomafont{subtitle}{\flushleft}
\addtokomafont{author}{\flushleft\setlength{\tabcolsep}{0pt}}
\addtokomafont{date}{\flushleft}
\addtokomafont{publishers}{\flushleft}

\titlehead{\includegraphics[scale=2]{logo_en}}
\subject{Software Engineering and Design}
\title{SE Process}
\subtitle{Mental Health Care Patient Management System (MHC-PMS)}
\author{
\begin{tabular}{l}
\normalfont\bfseries{Team White:}\\
Dellsperger Jan\\
Ellenberger Roger\\
Sheppard David\\
Sidler Matthias\\
Spring Mathias\\
Thöni Stefan
\end{tabular}
}
\date{\today}
\publishers{Version 1.0}

\begin{document}

\begin{titlepage}
	\maketitle
\end{titlepage}

\section{Introduction}
This document is a protocol of the discussion held about the SE process to be used for the MHC-PMS project.

\section{Approach}
The decision that has to be made is wether the project should be primarily plan-driven or if a more agile approach is the better option. Here are the two options as well as their respective pro and contra points.

\subsection{Plan-Driven}
The main advantage of the plan-driven approach are clear defined project goals and milestones. Furthermore it is easier to guess project costs if tasks are known up-front.

Disadvantages can be inflexibility to changes. Furthermore, it is possible that, if the client is not very involved, feedback can be given to late or not at all.

\subsection{Agile}
The agile approach gives the development team a lot of flexibility as well as fast customer feedback and quick recognition of possible problems.

The administrative overhead and the additional workload for the customer can however be disadvantageous.

\subsection{Conclusion}
The above discussion lead to the conclusion that a mixed approach is the way to go. The various stakeholders would make a purely agile approach very difficult whereas a plan-driven approach would leave to little flexibility for possible changes.

We will begin with a plan-driven approach for the initial phase of the project. As soon as the tasks are defined we will switch to an agile approach for the development phase.

\section{Activities}
The following are the things that have to be done before, during and after the project.

\subsection{Requirements Analysis}
The goal of the requirements analysis is to define what features the project must include in order for it to be able to go live. It is prepared in a brain-storming session by the development team and the client. Finally, a document is created that specifies the required features.

\subsection{Project Definition}
Deadlines, a general plan and development iterations are defined by the development team and written down in a document.

\subsection{Sprint-Planning}
The sprint-planning consists of a review of the previous sprint, the decision of what userstories are relevant for the next sprint  and the technical definition and time requirements of relevant userstories. It is conducted by the development team and the client.

\subsection{Sprint}
During the sprint the development team implements and tests the userstories and holds periodic meetings to discuss encountered problems and the general state of userstories and the project in general.

\end{document}
